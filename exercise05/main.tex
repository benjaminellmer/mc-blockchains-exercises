\documentclass[14pt,a4paper]{extarticle}
\usepackage{graphicx}
\usepackage{caption} 
\usepackage{hyperref}
\usepackage[T1]{fontenc}
\usepackage[utf8x]{inputenc}
\usepackage{pdfpages}
\usepackage{libertine}
\usepackage{listings}
\usepackage{xcolor}
\usepackage{enumitem}
\usepackage{fullpage}
\usepackage{ulem}
\usepackage{cancel}

\renewcommand{\thesection}{\arabic{section}}
\renewcommand{\familydefault}{\sfdefault}


\colorlet{punct}{red!60!black}
\definecolor{delim}{RGB}{20,105,176}
\colorlet{numb}{magenta!60!black}

\hypersetup{
    colorlinks=true,
    linkcolor=blue,
    filecolor=blue,      
    urlcolor=blue,
    pdfborder={0 0 0},
    linktocpage      
}

\begin{document}
	\begin{titlepage}
		\centering
		{\scshape\LARGE Blockchains \par}
		\vspace{2.5cm}
		{\huge\bfseries Giving Answers to Videos}
		\vfill
		{\normalsize von\par}
		{\normalsize Benjamin Ellmer (\textsc{S2210455012}) \par}
		\vspace{1cm}
		\includegraphics[width=0.3\textheight]{images/logo.pdf} \par
		\vspace{1cm}
		{\large Mobile Computing Master \par}
		{\large FH Hagenberg \par}
		\vfill
		{\large \today\par}
	\end{titlepage}

	\section*{Video 1: Line goes up - The problem with NFT}
	\subsection*{Was versteht man unter dem "fundamental cash problem"?}
	Das Problem ist, dass der Preis der Coins nur spekulativ ist, man kann sich mit seinen Crypotcurrencies nichts kaufen.
	Man muss also jemand anderes finden, der einem die Coins abkauft, oder der einem etwas dafür gibt.
	Es ist ein \textbf{Bigger Fool Scam} - man kann kein Geld mit Kryptowährung genrieren, jeder Dollar den man gewinnt wird von jemand anderes verloren.

	\subsection*{Beschreiben sie das Grundproblem des sogenannten "root proof of authenticity":}
	Die root-proof-of-authenticity bestätigt, ob jemand witklich der Ersteller z.B. eines Kunstwerks ist.
	Bei NFTs gibt es keine root-proof-of-authenticity, man kann sich auch einen NFT kaufen und genau diesen NFT dann selbst minten.
	Ein System, indem man nicht garantieren kann wer der Ersteller ist, ist nicht vertrauenswürdig und hat keine Zukunft.

	\subsection*{Formulieren sie in eigenen Worten die im Video vorgestellte Kernaussage von David Gerard hinsichtlich NFTs:}
	Ich glaube er ist nicht der größte Fan von NFTs.
	NFTS sind nicht in der Lage, das was sie versprechen umzusetzen.
	Im Grunde genommen sind NTFs einfach Betrug, da sie nicht bestätigen, dass jemanden das was er gekauft hat wirklich gehört.
	Die einzigen Profiteure sind die Plattformen und diejenigen, die bereits Ether hatten, da der Preis so stark gestiegen ist.

	\pagebreak

	\section*{Video 2: Felix von Leitner - Was ist eigentlich Kryptowährung?}
	\subsection*{Warum wird Bitcoin als "Negative-sum" bezeichnet?}
	Innerhalb der Bitcoin Blockchain wird nur Bitcoin getauscht.
	Es wird also kein Geld geschaffen, sondern das Geld, das man verdient ist das Geld, das jemand anderes verliert.
	Bitcoin ist sogar schlechter, als ein Nullsummenspiel, da die miner Strom verbrauchen und um diesen zu bezahlen verkaufen sie Bitcoin.

	\subsection*{Was wird im Vortrag als "selbstverstärkende Oligarchie" bezeichnet?}
	Diejenigen, die bereits einen hohen Stake haben versgrössern ihren stake, da die Warhscheinlichkeit, den nächsten Block erstellen zu dürfen proportional zum stake ist.
	Somit sind die reichen die Oligarchen und die Oligarchen werden immer noch reicher (selbstverstärkend).
	Das Problem dabei ist, dass die Oligarchen zu Zentralisierung führen, wobei man doch Zentralisierung vermeiden wollte.

	\subsection*{Was wird laut dem Vortrag durch Smart Contracts "weg-optimiert"?}
	Laut dem Vortrag wird mittels Smart Contracts der Rechtsweg wegoptimiert, wovon wir eigentlich keinen Vorteil haben, da es den Rechtsweg für uns gibt.


\end{document}
